% Options for packages loaded elsewhere
\PassOptionsToPackage{unicode}{hyperref}
\PassOptionsToPackage{hyphens}{url}
%
\documentclass[
]{article}
\usepackage{amsmath,amssymb}
\usepackage{lmodern}
\usepackage{iftex}
\ifPDFTeX
  \usepackage[T1]{fontenc}
  \usepackage[utf8]{inputenc}
  \usepackage{textcomp} % provide euro and other symbols
\else % if luatex or xetex
  \usepackage{unicode-math}
  \defaultfontfeatures{Scale=MatchLowercase}
  \defaultfontfeatures[\rmfamily]{Ligatures=TeX,Scale=1}
\fi
% Use upquote if available, for straight quotes in verbatim environments
\IfFileExists{upquote.sty}{\usepackage{upquote}}{}
\IfFileExists{microtype.sty}{% use microtype if available
  \usepackage[]{microtype}
  \UseMicrotypeSet[protrusion]{basicmath} % disable protrusion for tt fonts
}{}
\makeatletter
\@ifundefined{KOMAClassName}{% if non-KOMA class
  \IfFileExists{parskip.sty}{%
    \usepackage{parskip}
  }{% else
    \setlength{\parindent}{0pt}
    \setlength{\parskip}{6pt plus 2pt minus 1pt}}
}{% if KOMA class
  \KOMAoptions{parskip=half}}
\makeatother
\usepackage{xcolor}
\usepackage[margin=1in]{geometry}
\usepackage{color}
\usepackage{fancyvrb}
\newcommand{\VerbBar}{|}
\newcommand{\VERB}{\Verb[commandchars=\\\{\}]}
\DefineVerbatimEnvironment{Highlighting}{Verbatim}{commandchars=\\\{\}}
% Add ',fontsize=\small' for more characters per line
\usepackage{framed}
\definecolor{shadecolor}{RGB}{248,248,248}
\newenvironment{Shaded}{\begin{snugshade}}{\end{snugshade}}
\newcommand{\AlertTok}[1]{\textcolor[rgb]{0.94,0.16,0.16}{#1}}
\newcommand{\AnnotationTok}[1]{\textcolor[rgb]{0.56,0.35,0.01}{\textbf{\textit{#1}}}}
\newcommand{\AttributeTok}[1]{\textcolor[rgb]{0.77,0.63,0.00}{#1}}
\newcommand{\BaseNTok}[1]{\textcolor[rgb]{0.00,0.00,0.81}{#1}}
\newcommand{\BuiltInTok}[1]{#1}
\newcommand{\CharTok}[1]{\textcolor[rgb]{0.31,0.60,0.02}{#1}}
\newcommand{\CommentTok}[1]{\textcolor[rgb]{0.56,0.35,0.01}{\textit{#1}}}
\newcommand{\CommentVarTok}[1]{\textcolor[rgb]{0.56,0.35,0.01}{\textbf{\textit{#1}}}}
\newcommand{\ConstantTok}[1]{\textcolor[rgb]{0.00,0.00,0.00}{#1}}
\newcommand{\ControlFlowTok}[1]{\textcolor[rgb]{0.13,0.29,0.53}{\textbf{#1}}}
\newcommand{\DataTypeTok}[1]{\textcolor[rgb]{0.13,0.29,0.53}{#1}}
\newcommand{\DecValTok}[1]{\textcolor[rgb]{0.00,0.00,0.81}{#1}}
\newcommand{\DocumentationTok}[1]{\textcolor[rgb]{0.56,0.35,0.01}{\textbf{\textit{#1}}}}
\newcommand{\ErrorTok}[1]{\textcolor[rgb]{0.64,0.00,0.00}{\textbf{#1}}}
\newcommand{\ExtensionTok}[1]{#1}
\newcommand{\FloatTok}[1]{\textcolor[rgb]{0.00,0.00,0.81}{#1}}
\newcommand{\FunctionTok}[1]{\textcolor[rgb]{0.00,0.00,0.00}{#1}}
\newcommand{\ImportTok}[1]{#1}
\newcommand{\InformationTok}[1]{\textcolor[rgb]{0.56,0.35,0.01}{\textbf{\textit{#1}}}}
\newcommand{\KeywordTok}[1]{\textcolor[rgb]{0.13,0.29,0.53}{\textbf{#1}}}
\newcommand{\NormalTok}[1]{#1}
\newcommand{\OperatorTok}[1]{\textcolor[rgb]{0.81,0.36,0.00}{\textbf{#1}}}
\newcommand{\OtherTok}[1]{\textcolor[rgb]{0.56,0.35,0.01}{#1}}
\newcommand{\PreprocessorTok}[1]{\textcolor[rgb]{0.56,0.35,0.01}{\textit{#1}}}
\newcommand{\RegionMarkerTok}[1]{#1}
\newcommand{\SpecialCharTok}[1]{\textcolor[rgb]{0.00,0.00,0.00}{#1}}
\newcommand{\SpecialStringTok}[1]{\textcolor[rgb]{0.31,0.60,0.02}{#1}}
\newcommand{\StringTok}[1]{\textcolor[rgb]{0.31,0.60,0.02}{#1}}
\newcommand{\VariableTok}[1]{\textcolor[rgb]{0.00,0.00,0.00}{#1}}
\newcommand{\VerbatimStringTok}[1]{\textcolor[rgb]{0.31,0.60,0.02}{#1}}
\newcommand{\WarningTok}[1]{\textcolor[rgb]{0.56,0.35,0.01}{\textbf{\textit{#1}}}}
\usepackage{graphicx}
\makeatletter
\def\maxwidth{\ifdim\Gin@nat@width>\linewidth\linewidth\else\Gin@nat@width\fi}
\def\maxheight{\ifdim\Gin@nat@height>\textheight\textheight\else\Gin@nat@height\fi}
\makeatother
% Scale images if necessary, so that they will not overflow the page
% margins by default, and it is still possible to overwrite the defaults
% using explicit options in \includegraphics[width, height, ...]{}
\setkeys{Gin}{width=\maxwidth,height=\maxheight,keepaspectratio}
% Set default figure placement to htbp
\makeatletter
\def\fps@figure{htbp}
\makeatother
\setlength{\emergencystretch}{3em} % prevent overfull lines
\providecommand{\tightlist}{%
  \setlength{\itemsep}{0pt}\setlength{\parskip}{0pt}}
\setcounter{secnumdepth}{-\maxdimen} % remove section numbering
\ifLuaTeX
  \usepackage{selnolig}  % disable illegal ligatures
\fi
\IfFileExists{bookmark.sty}{\usepackage{bookmark}}{\usepackage{hyperref}}
\IfFileExists{xurl.sty}{\usepackage{xurl}}{} % add URL line breaks if available
\urlstyle{same} % disable monospaced font for URLs
\hypersetup{
  pdftitle={Modelos con Coeficientes Variando},
  pdfauthor={Bladimir Valerio Morales Torrez},
  hidelinks,
  pdfcreator={LaTeX via pandoc}}

\title{Modelos con Coeficientes Variando}
\author{Bladimir Valerio Morales Torrez}
\date{Julio 2022}

\begin{document}
\maketitle

\hypertarget{introducciuxf3n}{%
\section{Introducción}\label{introducciuxf3n}}

Los modelos aditivos generalizados (modelos no paramétricos), propuestos
por @Hastie\_Tibshirani\_1986 dan a conocer una nueva clase de modelos
de regresión, extendiendo y flexibilizando el modelo de regresión
clásico, reemplazando la función lineal por una función aditiva suave y
no paramétricas, la cual es estimada por el algoritmo de puntuación
local (local scoring algorithm).\\
Es así que, @Hastie\_Tibshirani\_1993, proponen otra generalización al
modelo de regresión lineal clásico, denominados modelos de coeficientes
variando (Varying-Coefficient Models VCMs), los cuales contienen
regresores lineales pero permiten que sus coeficientes cambien
suavemente con el valor de otras variables, que se denominan
``modificadores de efecto''.

\hypertarget{marco-teuxf3rico}{%
\section{Marco Teórico}\label{marco-teuxf3rico}}

\hypertarget{el-modelo}{%
\subsection{El modelo}\label{el-modelo}}

Supongamos que se tiene una variable aleatoria \(Y\) cuya distribución
depende de un parámetro \(\eta\) que será el predictor lineal y se
relaciona con la media \(\mu=\mathbb{E}(Y)\) mediante la función enlace
\(\eta=g(\mu)\), el modelo lineal generalizado con coeficientes variando
se puede representar como: \begin{eqnarray}\label{modelo.general}
\eta_i=\beta_0+x_i^{1}\beta_1(t_{1_i})+\cdots+x_i^p\beta_p(t_{p_i})
\end{eqnarray} Donde los \(t_1,\cdots,t_p\) cambian los coeficientes de
las covariables \(x^1,\cdots,x^p\) a través de las funciones
\(\beta_1,\cdots,\beta_p\). La dependencia de \(\beta_j\) en \(t_j\),
con \(j=1,...,p\) implica un tipo de interacción entre \(t_j\) y
\(x^j\), donde la variable \(t_j\) puede no ser tan diferente a \(x^j\),
pero también \(t_j\) puede ser una variable especial como el tiempo.\\
Si se toma el modelo Gaussiano, donde \(g(\mu)=\mu\) y la variable
aleatoria \(Y\) tiene distribución normal con media \(\eta\), el modelo
(\ref{modelo.general}) es de la forma

\begin{eqnarray}\label{modelo.normal}
Y_i=\beta_0+x_i^{1}\beta_1(t_{1_i})+\cdots+x_i^p\beta_p(t_{p_i})+\varepsilon_i
\end{eqnarray}

donde \(\mathbb{E}(\varepsilon)=0\) y \(Var(\varepsilon)=\sigma^2\)

\hypertarget{casos-del-modelo}{%
\subsubsection{Casos del modelo}\label{casos-del-modelo}}

El modelo de coeficientes variando generaliza los siguientes tipos de
modelos: a) Si \(\beta_j(t_j)=\beta_j\) una función constante, entonces
ese término es lineal en \(x^j\). Si todos los términos son lineales se
reduce a un modelo lineal generalizado clásico.\\
b) Si \(x^j=c\) con \(c\) constante entonces el \(j-ésimo\) término es
\(\beta_j(t_j)\) una función no conocida. Si todos los términos tienen
esta característica se tiene un modelo aditivo generalizado.\\
c) Una función lineal \(\beta_j(t_j)=\beta_jt_j\) produce una
interacción de la forma \(\beta_jx^jt_j\).\\
d) En el término del modelo \(x\beta(t)\), cuando \(x\) es una variable
binaria (\(0-1\)). Supongamos que hay un término \(\beta_0(t)\) en el
modelo. Esto equivale a tener una curva separada correspondiente a cada
uno de los dos valores de \(x\).\\
e) A menudo las \(t_j\) serán la misma variable, como puede ser la edad
o el tiempo, que podría modificar los efectos de \(x^1,...,x^p\).
Supongamos que, los datos consisten en mediciones repetidas de las
variables \(y, x^1 , ..., x^p\) sobre \(n\) puntos de tiempo
\(t\in(t_1,...,t_n\) ) . Entonces podríamos modelar esto a lo largo del
tiempo
\[\eta_{t,i} =\beta_0(t_i) +x^1 (t_{1_i})\beta_1(t_{1_i}) + ... +x^p (t_{p_i})\beta_p(t_{p_i})\]
Que es llamado como \emph{modelo lineal generalizado dinámico} o
\emph{condicionalmente paramétricos}\\
f) La variable modificante \(t_j\) puede ser la misma que \(x^j\), por
simplicidad suponemos que es un modelo lineal normal con un solo
término. \[y_i =x_i \beta_j(x_i) + \varepsilon\] Este es un modelo común
de suavizamiento o regresión no paramétrica de \(Y\) versus \(X\).\\
g) Cada \(t_j\) puede tener un valor escalar o vectorial.\\
h) En todos los casos anteriores, hay muchas formas de modelar las
funciones \(\beta_j(t_j)\). Por ejemplo, podríamos usar representaciones
paramétricas flexibles tales como polinomios, series de Fourier o
polinomios por partes, o de manera más general funciones no
paramétricas, mediante el uso de métodos kernel, penalización o
formulaciones bayesianas estocásticas.

\hypertarget{matricialmente}{%
\subsubsection{Matricialmente}\label{matricialmente}}

\hypertarget{estimaciuxf3n}{%
\subsection{Estimación}\label{estimaciuxf3n}}

El modelo (\ref{modelo.general}), es general para la mayoría de las
aplicaciones, ya que no se imponen restricciones a las funciones de los
coeficientes. La estimación no paramétrica sin restricciones de estas
funciones probablemente no sería posible, salvo en el caso de diseños
especiales. En el caso de los datos observacionales, es probable que se
vea un valor diferente para cada \(t\), en cada muestra.\\
Por esta razón, se impone restricciones a las funciones de los
coeficientes variando, por ejemplo, a trozos con forma paramétrica
conocida o bien suave pero no paramétrica. Una aproximación sería a
través de bases paramétricas como funciones polinómicas o
trigonométricas. Normalmente, éstas no proporcionan suficiente
flexibilidad y adaptabilidad local, y es probable que un conjunto de
bases de splines de regresión es preferible.\\
Se procederá igual que con un modelo lineal, sólo que con varias
variables definidas por los productos de cada \(x\), y las bases de
\(\beta_j(t_j)\) con este enfoque, las herramientas inferenciales
estándar sirven para evaluar conjuntos de coeficientes, puntos de
influencia, etc. Las características de las curvas ajustadas pueden ser
muy diferentes con pequeños cambios en las posiciones de los nudos,
sobre todo si sólo se pueden permitir unos pocos.\\
En este trabajo se mostrará el procedimiento para el modelo
(\ref{modelo.normal}) ya que para el modelo (\ref{modelo.general})
incluyen procedimientos diferentes como el algoritmo de tipo
NewtonRaphson.

\hypertarget{muxednimos-cuadrados-penalizados}{%
\subsubsection{Mínimos Cuadrados
Penalizados}\label{muxednimos-cuadrados-penalizados}}

Se tiene
\[\mathbb{E}\left\{y_i-\sum_{j=1}^{p}x_i^j \beta_j(t_{j_i})\right\}^2\]
Se condiciona en \(t_j\), es una condición suficiente para la solución:

\[\mathbb{E}\left\{x^k_i\left\{y_i-\sum_{i=1\\i\neq k}^{p} x_i^j \beta_j(t_{j_i}) \right\}|t_k\right\}=0 \hspace{1cm}k=1,...,p\]

La ecuación (6) es una relación de dos esperanzas condicionales y puede
verse como una media ponderada condicional, en la que las ponderaciones
condicionales vienen dadas por el término XJ; el término en el
denominador garantiza que las ponderaciones se integren en 1. Existe una
ecuación ecuación similar para cada función (3i'' j = 1, \ldots, p, y el
conjunto de p ecuaciones debe resolver simultáneamente para la función
(3i'' j = 1 \ldots, p, como una estimación flexible de una esperanza
condicional, esto sugiere que cada función (3j puede estimarse de forma
iterativa ``de uno en uno'' mediante el alisado \{Y-I:ki
``jXk(3k(Rk)\}IXj en R j, con pesos XJ. Esta es la idea central del
marco formal que se discute a continuación

\hypertarget{aplicaciuxf3n}{%
\section{Aplicación}\label{aplicaciuxf3n}}

Se utilizará los datos de los rendimientos diarios de las acciones de la
General Electric Company y el índice de Standard \& Poor's 500 (S\&P
500). Los están en el \texttt{data.frame} \emph{capm} en el paquete
\texttt{HRW}. La variable \texttt{Date} se tiene desde el 1 de noviembre
de 1993 al 31 de marzo de 2003, por lo que hay más de nueve años de
datos. Si \(P_t\) es el precio de una acción en el día \(t\), entonces
el rendimiento logarítmico de ese día es \(log(\frac{P_t}{P_{t−1}} )\).
El exceso de rendimiento logarítmico es el rendimiento logarítmico menos
la tasa de interés libre de riesgo, generalmente considerada como la
tasa de facturación del Tesoro a corto plazo.\\
El siguiente código calcula el rendimiento logarítmico excedente tanto
de las acciones de General Electric Company como del índice S\&P 500:

\begin{Shaded}
\begin{Highlighting}[]
\FunctionTok{library}\NormalTok{(HRW) }
\end{Highlighting}
\end{Shaded}

\begin{verbatim}
## Warning: package 'HRW' was built under R version 4.1.3
\end{verbatim}

\begin{Shaded}
\begin{Highlighting}[]
\FunctionTok{data}\NormalTok{(capm)}
\NormalTok{n }\OtherTok{\textless{}{-}} \FunctionTok{dim}\NormalTok{(capm)[}\DecValTok{1}\NormalTok{]}
\CommentTok{\#riesgo.libre=riskfree}
\NormalTok{riesgo.libre }\OtherTok{\textless{}{-}}\NormalTok{ capm}\SpecialCharTok{$}\NormalTok{Close.tbill[}\DecValTok{2}\SpecialCharTok{:}\NormalTok{n]}\SpecialCharTok{/}\NormalTok{(}\DecValTok{100}\SpecialCharTok{*}\DecValTok{365}\NormalTok{)}
\CommentTok{\#elrGE=rend.log.ex.GE}
\NormalTok{rend.log.ex.GE }\OtherTok{\textless{}{-}} \FunctionTok{diff}\NormalTok{(}\FunctionTok{log}\NormalTok{(capm}\SpecialCharTok{$}\NormalTok{Close.ge))}\SpecialCharTok{{-}}\NormalTok{riesgo.libre}
\NormalTok{rend.log.ex.SP500 }\OtherTok{\textless{}{-}} \FunctionTok{diff}\NormalTok{(}\FunctionTok{log}\NormalTok{(capm}\SpecialCharTok{$}\NormalTok{Close.sp500))}\SpecialCharTok{{-}}\NormalTok{riesgo.libre}

\FunctionTok{plot}\NormalTok{(rend.log.ex.GE)}
\end{Highlighting}
\end{Shaded}

\includegraphics{doc_CVM_files/figure-latex/unnamed-chunk-1-1.pdf}

\begin{Shaded}
\begin{Highlighting}[]
\FunctionTok{plot}\NormalTok{(rend.log.ex.SP500)}
\end{Highlighting}
\end{Shaded}

\includegraphics{doc_CVM_files/figure-latex/unnamed-chunk-1-2.pdf}

\begin{Shaded}
\begin{Highlighting}[]
\FunctionTok{hist}\NormalTok{(rend.log.ex.GE,}\AttributeTok{freq =} \ConstantTok{FALSE}\NormalTok{)}
\end{Highlighting}
\end{Shaded}

\includegraphics{doc_CVM_files/figure-latex/unnamed-chunk-1-3.pdf}

\begin{Shaded}
\begin{Highlighting}[]
\FunctionTok{hist}\NormalTok{(rend.log.ex.SP500,}\AttributeTok{freq =} \ConstantTok{FALSE}\NormalTok{)}
\end{Highlighting}
\end{Shaded}

\includegraphics{doc_CVM_files/figure-latex/unnamed-chunk-1-4.pdf}

\begin{Shaded}
\begin{Highlighting}[]
\FunctionTok{plot}\NormalTok{(rend.log.ex.GE,rend.log.ex.SP500)}
\end{Highlighting}
\end{Shaded}

\includegraphics{doc_CVM_files/figure-latex/unnamed-chunk-1-5.pdf}

Donde \texttt{capm\$Close.sp500} y \texttt{capm\$Close.ge} son los
precios de cierre diarios y \texttt{capm\$Close.tbill} es la tasa diaria
de facturación del Tesoro expresada como porcentaje. En el código,
\texttt{capm\$Close.tbill} primero se divide por \(100\) para convertir
de un porcentaje a una fracción y luego se divide por \(365\) para
convertir de una tasa anual a una tasa diaria. Por eso,
\texttt{rend.log.ex.GE} y \texttt{rend.log.ex.SP500} son los
rendimientos logarítmicos excedentes diarios de las acciones de General
Electric y en el índice S\&P 500, respectivamente.\\
Uno de los supuestos del modelo de fijación de precios de activos de
capital en finanzas es que los rendimientos logarítmicos excedentes
medios de una acción dependen linealmente de los rendimientos
logarítmicos excedentes del mercado. El rendimiento del índice S\&P 500
suele sera utilizado como proxy de la rentabilidad del mercado. Por lo
general, el modelo de regresión lineal simple, se ajusta solo a datos
recientes, ya que se espera que la pendiente cambie lentamente a lo
largo del tiempo. Una alternativa es ajustar un modelo de coeficiente
variando a todos los datos, con el intercepto y pendiente dependiendo de
la fecha. Hacer esto proporcionaría evidencia como y con qué rapidez
está cambiando la pendiente. La teoría económica predice que el
intercepto es cero, por lo que se espera que el intercepto estimado sea
insignificante. Primero, ajustamos un modelo de regresión lineal simple
a todos los datos usando:

\begin{Shaded}
\begin{Highlighting}[]
\FunctionTok{library}\NormalTok{(mgcv)}
\end{Highlighting}
\end{Shaded}

\begin{verbatim}
## Warning: package 'mgcv' was built under R version 4.1.3
\end{verbatim}

\begin{verbatim}
## Loading required package: nlme
\end{verbatim}

\begin{verbatim}
## This is mgcv 1.8-40. For overview type 'help("mgcv-package")'.
\end{verbatim}

\begin{Shaded}
\begin{Highlighting}[]
\NormalTok{fitSLR }\OtherTok{\textless{}{-}} \FunctionTok{gam}\NormalTok{(rend.log.ex.GE }\SpecialCharTok{\textasciitilde{}}\NormalTok{ rend.log.ex.SP500)}
\FunctionTok{summary}\NormalTok{(fitSLR)}
\end{Highlighting}
\end{Shaded}

\begin{verbatim}
## 
## Family: gaussian 
## Link function: identity 
## 
## Formula:
## rend.log.ex.GE ~ rend.log.ex.SP500
## 
## Parametric coefficients:
##                    Estimate Std. Error t value Pr(>|t|)    
## (Intercept)       0.0002971  0.0002628   1.131    0.258    
## rend.log.ex.SP500 1.2441346  0.0227404  54.710   <2e-16 ***
## ---
## Signif. codes:  0 '***' 0.001 '**' 0.01 '*' 0.05 '.' 0.1 ' ' 1
## 
## 
## R-sq.(adj) =  0.559   Deviance explained = 55.9%
## GCV = 0.00016324  Scale est. = 0.0001631  n = 2362
\end{verbatim}

Para verificar si un modelo lineal es apropiado, ajustamos el modelo de
regresión no paramétrico del rendimiento logarítmico excedente de las
acciones de General Electric que es una función suave e invariable en el
tiempo del rendimiento logarítmico excedente del índice S\&P 500. este
modelo en comparación con el modelo lineal simple a través de una prueba
\(F\).

\begin{Shaded}
\begin{Highlighting}[]
\NormalTok{fitNPR }\OtherTok{\textless{}{-}} \FunctionTok{gam}\NormalTok{(rend.log.ex.GE }\SpecialCharTok{\textasciitilde{}} \FunctionTok{s}\NormalTok{(rend.log.ex.SP500),}\AttributeTok{method =} \StringTok{"REML"}\NormalTok{)}
\FunctionTok{summary}\NormalTok{(fitNPR)}
\end{Highlighting}
\end{Shaded}

\begin{verbatim}
## 
## Family: gaussian 
## Link function: identity 
## 
## Formula:
## rend.log.ex.GE ~ s(rend.log.ex.SP500)
## 
## Parametric coefficients:
##              Estimate Std. Error t value Pr(>|t|)  
## (Intercept) 0.0004786  0.0002612   1.832   0.0671 .
## ---
## Signif. codes:  0 '***' 0.001 '**' 0.01 '*' 0.05 '.' 0.1 ' ' 1
## 
## Approximate significance of smooth terms:
##                        edf Ref.df     F p-value    
## s(rend.log.ex.SP500) 6.919  8.032 380.5  <2e-16 ***
## ---
## Signif. codes:  0 '***' 0.001 '**' 0.01 '*' 0.05 '.' 0.1 ' ' 1
## 
## R-sq.(adj) =  0.564   Deviance explained = 56.5%
## -REML = -6940.1  Scale est. = 0.00016116  n = 2362
\end{verbatim}

\begin{Shaded}
\begin{Highlighting}[]
\FunctionTok{anova}\NormalTok{(fitSLR,fitNPR,}\AttributeTok{test =} \StringTok{"F"}\NormalTok{)}
\end{Highlighting}
\end{Shaded}

\begin{verbatim}
## Analysis of Deviance Table
## 
## Model 1: rend.log.ex.GE ~ rend.log.ex.SP500
## Model 2: rend.log.ex.GE ~ s(rend.log.ex.SP500)
##   Resid. Df Resid. Dev     Df  Deviance      F    Pr(>F)    
## 1    2360.0    0.38492                                      
## 2    2352.2    0.37939 7.7822 0.0055265 4.4064 3.272e-05 ***
## ---
## Signif. codes:  0 '***' 0.001 '**' 0.01 '*' 0.05 '.' 0.1 ' ' 1
\end{verbatim}

La prueba \(F\) rechaza el modelo de regresión lineal simple. Sin
embargo, el tamaño de la muestra es bastante grande ya que hubo \(2362\)
días de retornos y por lo tanto, hay un poder sustancial para detectar
desviaciones de la linealidad.

La gráfica de la regresión no paramétrica fit(fitNPR) se desvía solo
ligeramente de una línea recta, al menos sobre el rango de la mayor
parte de los datos.

\begin{Shaded}
\begin{Highlighting}[]
\FunctionTok{plot}\NormalTok{(fitNPR)}
\end{Highlighting}
\end{Shaded}

\includegraphics{doc_CVM_files/figure-latex/unnamed-chunk-4-1.pdf}
Además, el ajuste suave tiene solo un poco más de \(R^2-ajustado\),
\(56,4\%\), en comparación con el del modelo lineal, \(55,9\%\). Cuando
recurrimos a modelos de coeficientes variando, como suposición de
trabajo supondremos que la regresión de los rendimientos logarítmicos
excedentes de General Electric sobre los rendimientos logarítmicos
excedentes del mercado es lineal, pero la intersección y la pendiente
varían con el tiempo.\\
Ahora, se ajusta y se grafica el modelo de coeficientes variando y lo
comparamos con el modelo lineal de coeficientes constantes mediante una
prueba \(F\). En el modelo de coeficientes variando, el intercepto y la
pendiente de la regresión lineal del logaritmo de retorno excedente de
General Electric con el excedente de rendimiento logarítmico del índice
S\&P 500 son funciones suaves en el tiempo (años), almacenada como la
variable \(t\):

\begin{Shaded}
\begin{Highlighting}[]
\NormalTok{dayNums }\OtherTok{\textless{}{-}}\NormalTok{ (}\DecValTok{1}\SpecialCharTok{:}\NormalTok{(n}\DecValTok{{-}1}\NormalTok{))}\SpecialCharTok{/}\NormalTok{(n}\DecValTok{{-}1}\NormalTok{)}
\NormalTok{startTime }\OtherTok{\textless{}{-}} \DecValTok{1993} \SpecialCharTok{+} \DecValTok{11}\SpecialCharTok{/}\DecValTok{12} \CommentTok{\#Noviembre}
\NormalTok{endTime }\OtherTok{\textless{}{-}} \DecValTok{2003} \SpecialCharTok{+} \DecValTok{3}\SpecialCharTok{/}\DecValTok{12} \CommentTok{\#Marzo}
\NormalTok{t }\OtherTok{\textless{}{-}}\NormalTok{ startTime }\SpecialCharTok{+}\NormalTok{ (endTime }\SpecialCharTok{{-}}\NormalTok{ startTime)}\SpecialCharTok{*}\NormalTok{dayNums}
\NormalTok{fitVCM }\OtherTok{\textless{}{-}} \FunctionTok{gam}\NormalTok{(rend.log.ex.GE }\SpecialCharTok{\textasciitilde{}} \FunctionTok{s}\NormalTok{(t) }\SpecialCharTok{+} \FunctionTok{s}\NormalTok{(t,}\AttributeTok{by =}\NormalTok{ rend.log.ex.SP500), }\AttributeTok{method =} \StringTok{"REML"}\NormalTok{)}
\FunctionTok{summary}\NormalTok{(fitVCM)}
\end{Highlighting}
\end{Shaded}

\begin{verbatim}
## 
## Family: gaussian 
## Link function: identity 
## 
## Formula:
## rend.log.ex.GE ~ s(t) + s(t, by = rend.log.ex.SP500)
## 
## Parametric coefficients:
##             Estimate Std. Error t value Pr(>|t|)
## (Intercept) 0.000331   0.000262   1.263    0.207
## 
## Approximate significance of smooth terms:
##                          edf Ref.df     F p-value    
## s(t)                   1.007  1.013   0.0       1    
## s(t):rend.log.ex.SP500 6.958  8.080 377.3  <2e-16 ***
## ---
## Signif. codes:  0 '***' 0.001 '**' 0.01 '*' 0.05 '.' 0.1 ' ' 1
## 
## R-sq.(adj) =  0.564   Deviance explained = 56.5%
## -REML = -6935.9  Scale est. = 0.00016129  n = 2362
\end{verbatim}

\begin{Shaded}
\begin{Highlighting}[]
\FunctionTok{anova}\NormalTok{(fitSLR,fitVCM,}\AttributeTok{test =} \StringTok{"F"}\NormalTok{)}
\end{Highlighting}
\end{Shaded}

\begin{verbatim}
## Analysis of Deviance Table
## 
## Model 1: rend.log.ex.GE ~ rend.log.ex.SP500
## Model 2: rend.log.ex.GE ~ s(t) + s(t, by = rend.log.ex.SP500)
##   Resid. Df Resid. Dev     Df  Deviance      F    Pr(>F)    
## 1    2360.0    0.38492                                      
## 2    2350.9    0.37953 9.0935 0.0053903 3.6751 0.0001303 ***
## ---
## Signif. codes:  0 '***' 0.001 '**' 0.01 '*' 0.05 '.' 0.1 ' ' 1
\end{verbatim}

Las funciones de intersección y pendiente se representan en la siguiente
gráfica. El argumento seleccionado de gam() se usa para seleccionar qué
componente trazar. El gráfico de código (fitVCM) sin el uso de
seleccionar trazaría tanto la intersección como la pendiente pero usaría
la misma escala en el eje y, lo cual no permitiría especificar valores
separados de \texttt{ylab} para las dos parcelas. El argumento
\texttt{seWithMean\ =\ TRUE} se usa para especificar que los intervalos
de confianza deben incluir la incertidumbre sobre la media.

\begin{Shaded}
\begin{Highlighting}[]
\FunctionTok{par}\NormalTok{(}\AttributeTok{mfrow=}\FunctionTok{c}\NormalTok{(}\DecValTok{1}\NormalTok{,}\DecValTok{2}\NormalTok{))}
\FunctionTok{plot}\NormalTok{(fitVCM,}\AttributeTok{select =} \DecValTok{1}\NormalTok{,}\AttributeTok{ylim =} \FunctionTok{c}\NormalTok{(}\SpecialCharTok{{-}}\FloatTok{0.001}\NormalTok{,}\FloatTok{0.001}\NormalTok{),}\AttributeTok{seWithMean =} \ConstantTok{TRUE}\NormalTok{)}
\FunctionTok{plot}\NormalTok{(fitVCM,}\AttributeTok{select =} \DecValTok{2}\NormalTok{,}\AttributeTok{ylim =} \FunctionTok{c}\NormalTok{(}\FloatTok{0.5}\NormalTok{,}\FloatTok{1.6}\NormalTok{),}\AttributeTok{seWithMean =} \ConstantTok{TRUE}\NormalTok{)}
\end{Highlighting}
\end{Shaded}

\includegraphics{doc_CVM_files/figure-latex/unnamed-chunk-6-1.pdf}

La estimación de la intersección sugiere que la intersección fue cero o
casi nula, durante todo el período, lo que concuerda con la teoría
económica. La pendiente estimada varía entre aproximadamente \(1,0\) y
\(1,4\) durante este período. El modelo de regresión lineal simple se
rechaza en favor del modelo de coeficientes variando, aunque el último
proporciona solo una pequeña mejora en el ajuste medido por el
\(R^2-ajustado\).

\end{document}
